\documentclass{article}
\usepackage{mathtools, amsfonts, xcolor}
\usepackage[parfill]{parskip}
\title{Sets Homework}
\author{Mohibullah Meer}
\date{October 26, 2023}
% paired delimiter
\DeclarePairedDelimiter{\floor}{\lfloor}{\rfloor}
\DeclarePairedDelimiter{\ceil}{\lceil}{\rceil}
\begin{document}
\maketitle
\section*{5.1. What are these sets?:}
\subsection*{5.1a: $\{\{2,4,6\}\cup\{6,4\}\}\cap\{4,6,8\}$}
\begin{align*}
  \{\{2,4,6\}\cup\{6,4\}\}&\cap\{4,6,8\}\\
  \{2,4,6\}&\cap\{4,6,8\}\\
  \{4,6\}
\end{align*}
\subsection*{5.1b: $\mathcal{P}(\{7,8,9\})-\mathcal{P}(\{7,9\})$}
\begin{align*}
  \mathcal{P}(\{7,8,9\})&-\mathcal{P}(\{7,9\})\\
  \{\phi,\{7\},\{8\},\{9\},\{7,8\},\{8,9\},\{7,9\},\{7,8,9\}&-\{\phi,\{7\},\{9\},\{7,9\}\}\\
  \{\{8\},\{7,8\},\{8,9\},\{7,8,9\}\}
\end{align*}
\subsection*{5.1c: $\mathcal{P}(\phi)$}
\begin{align*}
  \mathcal{P}(&\phi)\\
  \{&\phi\}
\end{align*}
\subsection*{5.1d: $\{1, 3, 5\}\times\{0\}\\$}
\begin{align*}
  \{1, 3, 5\}\times\{0\}\\
  \{(1,0),(3,0),(5,0)\}
\end{align*}
\subsection*{5.1e: $\{2,4,6\}\times\phi\\$}
\begin{align*}
  \{2,4,6\}\times\phi\\
  \phi
\end{align*}
\subsection*{5.1f:}
\begin{align*}
  \mathcal{P}(\{0\})&\times\mathcal{P}(\{1\})\\
  \{\phi,\{0\}\}&\times\{\phi,\{1\}\}\\
  \{(\phi,\phi),(\phi,\{1\})&,(\{0\},\phi),(\{0\},\{1\})\}
\end{align*}
\subsection*{5.1g:}
\begin{align*}
  \mathcal{P}(\mathcal{P}(\{2\}))\\
  \mathcal{P}(\{\phi,\{2\}\})\\
  \{\phi,\{\phi\},\{\{2\}\},\{\phi,\{2\}\}\}
\end{align*}
\section*{5.3.: Show that if $A$ is a finite set, then $|\mathcal{P}(A)|=2^{|A|}$.}
Base case---$\phi$:
\begin{align*}
  \mathcal{P}(\phi)&=\{\phi\}\\
  |\mathcal{P}(\phi)|&=2^{0}
\end{align*}
Induction hypothesis---Suppose that for any $|A|\geq0$, $|\mathcal{P}(A)|=2^{|A|}$.

Induction step---$|A'|=|A|+1$:

$A'$ contains all the elements in $A$, except for one additional member. Therefore for every member of $\mathcal{P}(A)$, there are $2^{|A|}$ distinct combinations of members that include this new element. As a result, the cardinality of the powerset of $A'$ is:
\begin{align*}
  |\mathcal{P}(A')|&=2^{|A|}+2^{|A|}\\
  |\mathcal{P}(A')|&=2^1*2^{|A|}\\
  |\mathcal{P}(A')|&=2^{|A|+1}\\
  |\mathcal{P}(A')|&=2^{|A'|}  
\end{align*}
\section*{5.5. Suppose:}
\subsection*{5.5a: Compare $|\mathcal{P}(A\times B)|$ and $|\mathcal{P}(A)|*|\mathcal{P}(B)|$. Under what circumstances is one larger than the other, and what is their ratio?}
Only when $|A|,|B|=0$ or $|A|,|B|=2$ will $|\mathcal{P}(A\times B)|=|\mathcal{P}(A)|*|\mathcal{P}(B)|$. Otherwise:
\begin{itemize}
\item If both have more than 2 members each, $|\mathcal{P}(A\times B)|$ will be higher.
  \item If one set has at most 1 member, $|\mathcal{P}(A)|*|\mathcal{P}(B)|$ will be higher.
  \end{itemize}
\subsection*{5.5b: Is it inevitably true that $(A-B)\cap(B-A)=\phi$?}
Yes. Since $A-B$ is $A$ without any elements also in $B$, and $B-A$ is the inverse, no elements exist within both of these sets.
\section*{5.7. Decide whether each statement is true or false and why:}
\subsection*{5.7a: $\phi=\{\phi\}$}
False. $\phi$ has a cardinality of 0 while $\{\phi\}$ has a cardinality of 1.
\subsection*{5.7b: $\phi=\{0\}$}
False. $\phi$ has a cardinality of 0 while $\{0\}$ has a carindality of 1.
\subsection*{5.7c: $|\phi|=0$}
True. $\phi$ is an empty set, meaning that it has a cardinality of 0.
\subsection*{5.7d: $|\mathcal{P}(\phi)|=0$}
False. $\mathcal{P}(\phi)$ results in $\{\phi\}$, a set of 1 member.
\subsection*{5.7e: $\phi\in\{\}$}
True. $\phi$ and $\{\}$ have the same cardinality, so, being equal, $\phi$ can be a subset of $\{\}$.
\subsection*{5.7f: $\phi=\{x\in\mathbb{N}:x\leq0$ and $x>0\}$}
False. This set contains all natural numbers, while $\phi$ contains no members.
\section*{5.9. Prove the following:}
\subsection*{5.9a: $A\cap(A\cup B)=A$}
\begin{align*}
  A\cap(A\cup B)&=A\\
  (A\cap A)\cup(A\cap B)&=A\\
  A\cup(A\cap B)&=A
\end{align*}
$A\cap B$ is a subset of $A$, meaning that its union with $A$ is equal to the set itself.
\subsection*{5.9b: $A-(B\cap C)=(A-B)\cup(A-C)$}
In $A-(B\cap C)$, the elements in $B$ and $C$ are removed from $A$, resulting in a set that has all elements of $A$ except for the ones in common with \textit{both} $B$ and $C$.

In $(A-B)\cup(A-C)$, a union is made between two sets: $A$, without any elements in common with $B$, and $A$, without any in common with $C$. In both sets, the intersection of $B$ and $C$ is excluded from $A$. As a result, the union is a set that has all element of $A$ except for the ones in common with \textit{both} $B$ and $C$.
\section*{5.11.: Defining $\langle x,y\rangle$ to be $\{x,\{x,y\}\}$, prove that $\langle x,y\rangle=\langle u,v\rangle$ iff $x=u$ and $y=v$.}
\begin{align*}
  \langle x,y\rangle&=\langle u,v\rangle\\
  \{x,\{x,y\}\}&=\langle u,v\rangle\\
  \{x,\{x,y\}\}&=\{u,\{u,v\}\}\\
  \{x,\{x,y\}\}&=\{x,\{x,y\}\}
\end{align*}
\end{document}
