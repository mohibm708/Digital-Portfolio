\documentclass{article}

\usepackage{mathtools, amsfonts, hyperref}
\usepackage[parfill]{parskip}

\title{Chapter 2 Homework}
\author{Mohibullah Meer}
\date{September 11, 2023}

% paired delimiter
\DeclarePairedDelimiter{\floor}{\lfloor}{\rfloor}
\DeclarePairedDelimiter{\ceil}{\lceil}{\rceil}

\begin{document}

\maketitle

\section*{2.1: Is $-1$ an odd integer, as we have defined the term? Why or why not?}

An odd integer is represented by $2k+1$, where $k\in\mathbb{Z}$. If we assume $-1$ is an odd integer:
\begin{align*}
  -1&=2k+1\\
  -2&=2k\\
  -1&=k
\end{align*}
Then $-1\in\mathbb{Z}$ proves that $-1$ is an odd integer.

\section*{2.3: Prove that the product of two odd numbers is an odd number.}

Assuming for contradiction that given odd integers $a$ and $b$, $ab$ is even:
\begin{align*}
  j,k,l&\in\mathbb{Z}\\
  a&=2k+1\\
  b&=2j+1\\
  ab&=2l\\
  (2k+1)(2j+1)&=2l\\
  4jk+2j+2k+1&=2l\\
  2jk+j+k+0.5&=l
\end{align*}
The 0.5 in the formula means that $l\not\in\mathbb{Z}$, a contradiction. Thus, the product of two odd numbers must always be an odd number.

\section*{2.5: Prove that $\sqrt[3]{2}$ is irrational.}

Assume for contradiction: $\sqrt[3]{2}$ is rational.

All rational numbers can be represented by the fully reduced fraction $\frac{a}{b}$, where at most one of the integers $a$ or $b$ can be even.
\begin{align*}
  \sqrt[3]{2}&=\frac{a}{b}\\
  \sqrt[3]{2}b&=a\\
  2b^3&=a^3
\end{align*}
At this point, it can be deduced that $2b^3$, being a multiple of 2, is always even. As a result, since raising an integer to any whole and positive power results in an integer of the same parity, then $2b^3=a^3$ also proves that $a$ is even and can be represented by $2k$, where $k\in\mathbb{Z}$.
\begin{align*}
  2k&=a\\
  2b^3&=(2k)^3\\
  2b^3&=8k^3\\
  b^3&=4k^3
\end{align*}
Similarly with $2b^3$, it can be assumed that $4k^3$, a multiple of 4, is also always even. Considering the same rule about integers and their parity when raised to a whole and positive power, the problem presents itself when $b^3=4k^3$ proves $b$ to be even. $a$ and $b$ both being even contradicts the assumption that $\frac{a}{b}$, representing $\sqrt[3]{2}$, is irreducible. Due to this contradiction, $\sqrt[3]{2}$ must be an irrational number.

\section*{2.7: Show that there is a seven-sided die; that is, a polyhedron with seven sides that is equally likely to fall on any one of its faces.}

The distribution of probabilities on a die is most influenced by how much surface area each face has. Given a rod of equilateral pentagonal ends, there is a ratio between the pentagon's side length and the rod's length where:
\begin{itemize}
  \item If the side length is too long, the 2 faces of the pentagon will be larger
  \item If the rod is too long, the 5 sides of the rod will be larger
\end{itemize}
This indicates that there is a point where the surface area (and probability) of the die will be evenly distributed between the 2 faces of the pentagon and the 5 faces of the rod.

Solved with help from: \url{https://rpg.stackexchange.com/a/123787}

\section*{2.9. Prove or provide a counterexample:}

\subsection*{2.9a: If $c$ and $d$ are perfect squares, then $cd$ is a perfect square.}

True. If $c$ and $d$ are perfect squares, then, respectively, the factors that make them up are $\sqrt{c}$ and $\sqrt{d}$. To ensure a lack of redundancy, though:
\begin{align*}
  a&=\sqrt{c},a\in\mathbb{Z}\\
  b&=\sqrt{d},b\in\mathbb{Z}\\
  cd&=a^2*b^2\\
  cd&=(aa)(bb)\\
  cd&=(ab)(ab)\\
  cd&=(ab)^2\\
  \sqrt{cd}&=ab
\end{align*}
As $\sqrt{cd}$ is $ab$, whose factors are integers, $cd$ must be a perfect square.

\subsection*{2.9b: If $cd$ is a perfect square and $c\neq d$, then c and d are perfect squares.}

False. The square roots of all perfect squares are integers, so given:
\begin{align*}
  cd&=36\\
  \sqrt{cd}&=\sqrt{36}=6\Rightarrow\sqrt{cd}\in\mathbb{Z}\\
  c&=3\\
  d&=12\\
  3*12&=36=cd\\
  \sqrt{c}&=\sqrt{3}\approx1.73\Rightarrow\sqrt{c}\not\in\mathbb{Z}\\
  \sqrt{d}&=\sqrt{12}\approx3.46\Rightarrow\sqrt{d}\not\in\mathbb{Z}
\end{align*}
As $\sqrt{c},\sqrt{d}\not\in\mathbb{Z}$, $c$ and $d$ are not always perfect squares.

\subsection*{2.9c: If $c$ and $d$ are perfect squares such that $c>d$, and $x^2=c$ and $y^2=d$, then $x>y$.}

False. All positive numbers have one positive and one negative square root, so given:
\begin{align*}
  c&=25\\
  d&=16\\
  x^2&=25\\
  y^2&=16\\
  x&=\sqrt{25}=\pm5\\
  y&=\sqrt{16}=\pm4
\end{align*}
Then even if $c>d$, the value of $x$ can be $-$5 while $y$ can be 4, meaning that $x<y$.

\section*{2.11: Critique the following "proof":} % i dont know if youll look at this mr ben but what you are about to read is the fruit of over 15 years of professional hating. enjoy

\begin{align*}
  x&>y\\
  x^2&>y^2\\
  x^2-y^2&>0\\
  (x+y)(x-y)&>0\\
  x+y&>0\\
  x&>-y
\end{align*}

\subsection*{2.11.1!: Line 2}

\begin{align*}
  x>y\;&{\not\Rightarrow}\;x^2>y^2\\
  x&=2\\
  y&=-4\\
  x=2&>-4=y\\
  x^2=4&<16=y^2
\end{align*}

\subsection*{2.11.2!: Line 3}

\begin{align*}
  x>y\;&{\not\Rightarrow}\;x^2-y^2>0\\
  x&=3\\
  y&=-3\\
  x=3&>-3=y\\
  x^2-y^2&=9-9=0
\end{align*}

\subsection*{2.11.3!: Line 4}

\begin{align*}
  x>y\;&{\not\Rightarrow}\;(x+y)(x-y)>0\\
  x&=5\\
  y&=-5\\
  x=5&>-5=y\\
  (x+y)(x-y)&=(5-5)(5+5)=0
\end{align*}

\subsection*{2.11.4!: Line 5}

\begin{align*}
  x>y\;&{\not\Rightarrow}\;x+y>0\\
  x&=7\\
  y&=-8\\
  x=7&>-8=y\\
  x+y&=7-8<0
\end{align*}

\subsection*{2.11.5!: Line 6}

\begin{align*}
  x>y\;&{\not\Rightarrow}\;x>-y\\
  x&=1\\
  y&=-2\\
  x=1&>-2=y\\
  x=1&<2=-y
\end{align*}

\section*{2.13: Write the following statements in terms of quantifiers and implications:}

\subsection*{2.13a: Every positive real number has two distinct square roots.}

\begin{equation*}
  \forall\; n\in\mathbb{R},n>0\Rightarrow\exists\;\sqrt{n},-\sqrt{n}:\sqrt{n}\neq-\sqrt{n}
\end{equation*}

\subsection*{2.13b: Every positive even number can be expressed as the sum of two prime numbers.}

\begin{equation*}
  \forall\;k\in\mathbb{N},k\neq0\Rightarrow\exists\;p_1,p_2\in\mathbb{P}:2k=p_1+p_2 
\end{equation*}

\section*{2.15: Using concepts developed in Chapter 1, explain that one "of the remaining 5 people, there must be at least 3 whom X knows, or at least 3 whom X does not know."}

In Chapter 1, the Extended Pigeonhole Principle is introduced as a formula that can be used to find the "lowest maximum" amount of members with the same mapping in a pigeonhole-ish situation.

In order to prove that the lowest maximum is 3, assume a set of people P and of relationships R:
\begin{align*}
  P&=\{1,\ldots,6\}\\
  R&=\{1, 2\}\\
  \ceil*{\frac{|P|}{|R|}}&=k+1\\
  \ceil*{\frac{6}{2}}&=k+1\\
  \ceil*{3}&=k+1\\
  3&=k+1
\end{align*}
Then, in the most evenly distributed case, the lowest maximum must be 3 people that X knows or does not know.

\end{document}
