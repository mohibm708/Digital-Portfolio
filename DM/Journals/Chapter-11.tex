\documentclass{article}
\usepackage{mathtools, amsfonts, hyperref}
\usepackage[dvipsnames, table]{xcolor}
\usepackage[parfill]{parskip}
\title{Chapter 11 Homework}
\author{Mohibullah Meer}
\date{January 29, 2024}
% paired delimiter
\DeclarePairedDelimiter{\floor}{\lfloor}{\rfloor}
\DeclarePairedDelimiter{\ceil}{\lceil}{\rceil}
\newcommand\true{\textcolor{ForestGreen}{\texttt{T}}}
\newcommand\false{\textcolor{red}{\texttt{F}}}
\begin{document}
\maketitle
\section*{11.1.:}
\subsection*{11.1a: Convert the binary numeral $111001$ to decimal.}
\[\begin{array}{|c||c|c|c|c|c|c|}
\hline
\cellcolor{black}&2^5&2^4&2^3&2^2&2^1&2^0\\
\hline
\#_2&1&1&1&0&0&1\\
\hline
\#_{10}&32&16&8&0&0&1\\
\hline
\end{array}\]
\begin{align*}
  111001_2&=32_{10}+16_{10}+8_{10}+1_{10}\\
  111001_2&=57_{10}
\end{align*}
\subsection*{11.1b: Convert the decimal numeral $87$ to binary.}
\[\begin{array}{|c|c||c|}
\hline
\#_{10}&\#_{10}/2&\textcolor{blue}{R}\\
\hline
87&0&1\\
43&1&1\\
21&2&1\\
10&3&0\\
5&4&1\\
2&5&0\\
1&6&1\\
\hline
\end{array}\]
\[87_{10}=1010111_2\]
\section*{11.3.}
\subsection*{11.3a:}
\subsection*{11.3b:}
\subsection*{11.3c:}
\subsection*{11.3d:}
\subsection*{11.3e:}
\section*{11.5. Let $p\downarrow q$ denote the "nor" operator:}
\subsection*{11.5a: Write the truth table for $p\downarrow q$.}
\[\begin{array}{|c|c||c|}
\hline
p&q&\textcolor{blue}{p\downarrow q}\\
\hline
\false&\false&\true\\
\false&\true&\false\\
\true&\false&\false\\
\true&\true&\false\\
\hline
\end{array}\]
\subsection*{11.5b: Using only the $\downarrow$ operator, write a formula equivalent to $\neg p$.}
\[p\downarrow p\Leftrightarrow\neg p\]
\[\begin{array}{|c||c|c|}
\hline
p&\textcolor{blue}{p\downarrow p}&\neg p\\
\hline
\false&\true&\true\\
\true&\false&\false\\
\hline
\end{array}\]
\subsection*{11.5c: Show that $\lor$ and $\land$ can also be expressed using just the $\downarrow$ operator.}
\[(p\downarrow q)\downarrow(p\downarrow q)\Leftrightarrow p\lor q\]
\[\begin{array}{|c|c||c|c||c|}
\hline
p&q&p\downarrow q&\textcolor{blue}{(p\downarrow q)\downarrow(p\downarrow q)}&p\lor q\\
\hline
\false&\false&\true&\false&\false\\
\false&\true&\false&\true&\true\\
\true&\false&\false&\true&\true\\
\true&\true&\false&\true&\true\\
\hline
\end{array}\]
\[(p\downarrow p)\downarrow(q\downarrow q)\Leftrightarrow p\land q\]
\[\begin{array}{|c|c||c|c|c||c|}
\hline
p&q&p\downarrow p&q\downarrow q&\textcolor{blue}{(p\downarrow p)\downarrow(q\downarrow q)}&p\land q\\
\hline
\false&\false&\true&\true&\false&\false\\
\false&\true&\true&\false&\false&\false\\
\true&\false&\false&\true&\false&\false\\
\true&\true&\false&\false&\true&\true\\
\hline
\end{array}\]
\section*{11.7. Write the logical formulas for the values of $Z_0$, $Z_1$, and $C_{out}$ of the two bit adder, in terms of the inputs $X_0$, $Y_0$, $X_1$, and $Y_1$.}
\begin{equation*}
  Z_0=X_0\oplus Y_0
\end{equation*}
\begin{equation*}
  Z_1=(X_0\land Y_0)\oplus(X_1\oplus Y_1)
\end{equation*}
\begin{equation*}
  C_{out}=(X_1\land Y_1)\lor((X_0\land Y_0)\land(X_1\lor Y_1))
\end{equation*}
\[\begin{array}{|c|c|c|c||c|c|c|}
\hline
X_0&Y_0&X_1&Y_1&Z_0&Z_1&C_{out}\\
\hline
\false&\false&\false&\false&\false&\false&\false\\
\false&\false&\false&\true&\false&\true&\false\\
\false&\false&\true&\false&\false&\true&\false\\
\false&\false&\true&\true&\false&\false&\true\\
\false&\true&\false&\false&\true&\false&\false\\
\false&\true&\false&\true&\true&\true&\false\\
\false&\true&\true&\false&\true&\true&\false\\
\false&\true&\true&\true&\true&\false&\true\\
\true&\false&\false&\false&\true&\false&\false\\
\true&\false&\false&\true&\true&\true&\false\\
\true&\false&\true&\false&\true&\true&\false\\
\true&\false&\true&\true&\true&\false&\true\\
\true&\true&\false&\false&\false&\true&\false\\
\true&\true&\false&\true&\false&\false&\true\\
\true&\true&\true&\false&\false&\false&\true\\
\true&\true&\true&\true&\false&\true&\true\\
\hline
\end{array}\]
\section*{11.9. In terms of a 7-segment display:}
\subsection*{11.9a: Write out the strokes that should be turned on to represent each of the remaining digits.}
\begin{center}\begin{tabular}{|c||c|c|c|c|c|c|c|}
    \hline
    \#&A&B&C&D&E&F&G\\
    \hline
    0&\true&\true&\true&\false&\true&\true&\true\\
    3&\true&\false&\true&\true&\false&\true&\true\\
    4&\false&\true&\true&\true&\false&\true&\false\\
    5&\true&\true&\false&\true&\false&\true&\true\\
    6&\true&\true&\false&\true&\true&\true&\true\\
    7&\true&\false&\true&\false&\false&\true&\false\\
    9&\true&\true&\true&\true&\false&\true&\true\\
    \hline
\end{tabular}\end{center}
\subsection*{11.9b: Write a truth table for the A segment and the binary representation of a number from 0-9.}
\[\begin{array}{|c|c|c|c||c|}
    \hline
    2^3&2^2&2^1&2^0&\textcolor{blue}{\text{A}}\\
    \hline
    \false&\false&\false&\false&\true\\
    \false&\false&\false&\true&\false\\
    \false&\false&\true&\false&\true\\
    \false&\false&\true&\true&\true\\
    \false&\true&\false&\false&\false\\
    \false&\true&\false&\true&\true\\
    \false&\true&\true&\false&\true\\
    \false&\true&\true&\true&\true\\
    \true&\false&\false&\false&\true\\
    \true&\false&\false&\true&\true\\
    \hline
\end{array}\]
\subsection*{11.9c: Write a DNF formula for A based on the truth table from part (b).}
\begin{gather*}
  (\neg p_0\land\neg p_1\land\neg p_2\land\neg p_3)\lor(\neg p_0\land\neg p_1\land p_2\land\neg p_3)\lor(\neg p_0\land\neg p_1\land p_2\land p_3)\\\lor(\neg p_0\land p_1\land\neg p_2\land p_3)\lor(\neg p_0\land p_1\land p_2\land\neg p_3)\lor(\neg p_0\land p_1\land p_2\land p_3)\\\lor(p_0\land\neg p_1\land\neg p_2\land\neg p_3)\lor(p_0\land\neg p_1\land\neg p_2\land p_3)
  \end{gather*}
\subsection*{11.9d: Draw a logic circuit that implements the formula from part (c).}
\textcolor{blue}{\underline{\href{http://tinyurl.com/yqdsyb2x}{7-segment 0-9 display.}}}
\section*{11.11. Regarding the Thue sequence:}
\subsection*{11.11a: Prove that $t_n$ is the exclusive or of the bits of the binary notation for $n$.}
\subsection*{11.11b: Show that for every $n\geq0$, $t_{2n}=t_n$, and $t_{2n+1}$ is the complement of $t_n$.}
\end{document}
